% In this file you should put the actual content of the blueprint.
% It will be used both by the web and the print version.
% It should *not* include the \begin{document}
%
% If you want to split the blueprint content into several files then
% the current file can be a simple sequence of \input. Otherwise It
% can start with a \section or \chapter for instance.


% In this file you should put the actual content of the blueprint.
% It will be used both by the web and the print version.
% It should *not* include the \begin{document}
%
% If you want to split the blueprint content into several files then
% the current file can be a simple sequence of \input. Otherwise It
% can start with a \section or \chapter for instance.

\title{Structure in Prime Gaps}

\date{\today}

\begin{abstract}
In this paper we formalize the results presented in the article "Structure in Prime Gaps".
\end{abstract}

\maketitle

\tableofcontents

\textbf{Acknowledgements}: We are grateful to everyone who has helped in the formalization of these results. 

\section{Introduction}
The goal of this document is to provide a formalized version of various mathematical claims, definitions, and proofs from the paper "Structure in Prime Gaps" using Lean 4. This paper will then serve as a blueprint for the resulting code written in Lean 4.

\section{The Cayley Table T}

\section{The Sub-Array: $TT_i$}

%Definition 2: Define a v x w sub-array TTi of T such that v, w ≥ 2.

\section{The 4-tuple $\beta$ = (A, B, L, E)}

%Definition 3: Define a 4-tuple β = (A, B, L, E) such that the values of its elements are the vertices of TTi.

\section{Lemma 4.1}

%Lemma 4.1 : Prove that: given a 4-tuple TTi.β then TTi.β.A + TTi.β.E = TTi.β.B + TTi.β.L.

\section{Lemma 4.2}

%Lema 4.2 : Prove that: All prime numbers greater than 3 can be expressed in the form 6n + 1 or 6n − 1.

\section{Lemma 4.3}

%Let tm,n be a term in T where the indexes m and n are zero based and refer to the rows and columns in T respectively.

%Prove that: 

%For every prime pα ≥ 5, there exists a sub-array TTi ∈ T such that the following properties are simultaneously true;

%Property 1 : TTi.β.A + 3 ∈ {6n ± 1|n ∈ N1}

%Property 2 : (TTi.β.B + 3) − TTi.β.E ∈ {6n ± 1|n ∈ N1}

%Property 3 : TTi.β.L ≡ 0 (mod 6)

%Property 4 : TTi.β.A = TTi.β.E

%Property 5 : TTi.β.B + 3 ∈ {6n ± 1|n ∈ N1}

%If and only if

%for TTi.β.A + 3 ∈ {6n − 1|n ∈ N1};

%TTi.β.A = 6x + 6y − 4 

%TTi.β.B = 6x + 12y − 8 

%TTi.β.L = 6x

%TTi.β.E = 6x + 6y − 4

%for TTi.β.A + 3 ∈ {6n + 1|n ∈ N1};

%TTi.β.A = 6x + 6y − 2 

%TTi.β.B = 6x + 12y − 4 

%TTi.β.L = 6x 

%TTi.β.E = 6x + 6y − 2

%where n ∈ N1, x < n, y > 0, n = x + y, (TTi.β.A + 3) = pα and TTi.β.A = t2,k.

\section{Lemma 4.4}

%Lemma 4.4: Let any sub-array TTi that satisfies Lemma 4.3 be referred to as a Prime Array.

%Prove that: For every prime pα ≥ 5, there are infinitely many Prime Arrays such that TTi.β.A = pα − 3.

\section{Lemma 4.5}

%Lemma 4.5: Prove that: For every prime pα ≥ 5, there exists infinitely many Prime Arrays, TTi, such that TTi.β.A = pα − 3 %and (Ti.β.B + 3) and ((Ti.β.B + 3) −Ti.β.E) are prime.

\section{Theorem 1}

%Theorem 1: Prove that: For every prime pα, there exists infinitely many pairs of primes, (pn, pn+m), such that (pn+m − pn) = %pα − 3, where n, α ≥ 3, m ≥ 1, and pn is the nth prime.

\section{Theorem 2}

%Theorem 2: Prove that there exist infinitely many pairs of primes with a gap of 2.


%\printbibliography

